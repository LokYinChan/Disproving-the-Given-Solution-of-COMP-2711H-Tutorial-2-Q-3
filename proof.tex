\documentclass[a4paper,10pt]{article}
\usepackage[margin=1in]{geometry}
\usepackage{comment}  % Load comment package FIRST
\usepackage{amsmath}
\usepackage{amssymb}
\usepackage{amsfonts}
\usepackage[T1]{fontenc}
\usepackage{mathtools}
\usepackage{graphicx}

\DeclarePairedDelimiter\floor{\lfloor}{\rfloor}
\DeclarePairedDelimiter\ceil{\lceil}{\rceil}


\begin{document}
    This document is based on Amir's 2711h tutorial 2 Question 3, whose solution I disagree with. The following is written to find the general solution to determine which player must win. 

    $\\$

    Given two stacks of blocks. Name the one with fewer blocks as $A$ and another as $B$. 
    Let the number of blocks that $A$ and $B$ have be $a$ and $b$ respectively.

    Let player M be the player first moves and another player N.
    
    Hypothesis: if there exists $d \in \mathbb{N}_{\geq 1}$ such that $a$ and $b$ can be expressed in form of $1 + d$ and $1 + 2d$, then the play must lose, else must win.
    $\\ \\$
    Lemma 1: player facing $a = b \geq 1$ must win.

    Proof: The player can reduce $(a, b)$ to $(0, 0)$ easily.

    Lemma 2: The player facing $a = 0, b\geq 1$ must win.

    Proof: The player can reduce $(0, b)$ to $(0, 0)$ easily.

    $\\ \\$
    Base case: when $d = 1$:

    When player M faces $(1, 2)$, he can have the following moves only:

    $$\begin{cases}
        (1 , 2) \rightarrow (1, 1) & \text{loses according to Lemma 1} \\
        (1, 2) \rightarrow (1, 0) & \text{loses according to Lemma 2} \\
        (1, 2) \rightarrow (2, 0) & \text{loses according to Lemma 2}
    \end{cases}$$

    Note that if $b < 2$, the situation becomes $(0,1)$ or $(1, 1)$, then player M must win


    if $b > 2$, then player M can $(1, b) \rightarrow (1, 2)$, which means player N must lose.

    $\\ \\$
    Induction steps: Assume there exists $k \in \mathbb{N}$ such that the statement is true for all $i \in \mathbb{N}_{\leq k}$, ie player M must lose if he faces $(1 + i, 1 + 2i)$. Consider $k + 1$:
    
    If player M faces $(1 + k + 1, 1 + 2k + 2)$, he can have the following move:

    $$\\ \\$$
    1. Removing $j \leq k$ blocks from A: 
        $$(1 + k + 1, 1 + 2k + 2) \rightarrow (1 + k + 1 - j, 1 + 2k + 2)$$ 
        Then player N can remove $2j$ blocks from B:
        $$(1 + k + 1 - j, 1 + 2k + 2) \rightarrow (1 + k + 1 - j, 1 + 2(k + 1 - j))$$
        where player M must lose by assumption.

    $$\\ \\$$
    2. Removing $j \leq k$ blocks from B: 
        $$(1 + k + 1, 1 + 2k + 2) \rightarrow (1 + k + 1, 1 + 2k + 2 - j)$$ 
        Then player N can remove $j$ blocks from both A and B:
        $$(1 + k + 1, 1 + 2k + 2 - j) \rightarrow (1 + k + 1 - j, 1 + 2(k + 1 - j))$$
        where player M must lose by assumption.
    \newpage

    3. Removing $j \leq k$ blocks from both A and B: 
        $$(1 + k + 1, 1 + 2k + 2) \rightarrow (1 + k + 1 - j, 1 + 2k + 2 - j)$$ 
        Then player N can remove $j$ blocks from  B:
        $$(1 + k + 1 - j, 1 + 2k + 2 - j) \rightarrow (1 + k + 1 - j, 1 + 2(k + 1 - j))$$
        where player M must lose by assumption.

    $$\\ \\$$
    4. Removing $j: k + 1 < j < 2k + 2$ blocks from B: 
        $$(1 + k + 1, 1 + 2k + 2) \rightarrow (1 + k + 1, 1 + 2k + 2 - j)$$ 
        Then player N can remove $3k + 3 - j$ blocks from both A and B:
        $$(1 + k + 1, 1 + 2k + 2 - j) \rightarrow (1 - 2k - 2 + 2j , 1 - k - 1 + j) = (1 + j - k - 1, 1 + 2(j - k - 1 ))$$
        where player M must lose by assumption.

    $$\\ \\$$
    5. Removing $k + 1, k + 2$ blocks from A, $2k + 2, 2k + 3$ blocks from B,
    
    or $k + 1, k + 2$ blocks from both A and B:
        
    These will result in $(0, n), (1, n)$, which must lose by Lemma 1 $\&$ 2.
    
    $$\\ \\$$
    If player M faces $(1 + k + 1, 1 + 2k + 2 - j)$ or $(1 + k + 1, 1 + 2k + 2 + j)$, where $1 \leq j \leq k$, he can do the following to make sure he can win:
    $$\begin{cases}
        \text{Remove $j$ blocks from both A and B} & (1 + k + 1, 1 + 2k + 2 - j)\rightarrow (1 + k + 1 - j, 1 + 2(k + 1 - j))\\
        \text{Remove $j$ blocks from  and B} & (1 + k + 1, 1 + 2k + 2 + j)\rightarrow (1 + k + 1, 1 + 2(k + 1 ))\\

    \end{cases}$$

    Therefore, by strong induction, we have successfully prove that if there exists $d \in \mathbb{N}_{\geq 1}$ such that $a$ and $b$ can be expressed in form of $1 + d$ and $1 + 2d$, then the play must lose, else must win.



\end{document}
